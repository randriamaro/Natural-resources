\documentclass[12pt]{article}

\usepackage{outlines}
\usepackage{enumitem}
\usepackage{alltt}%
\usepackage{csquotes}
\usepackage{anysize}
\usepackage{amsmath}
\usepackage{mathtools}
\usepackage{bbm}
\usepackage{booktabs}
\usepackage{tabularx}
\usepackage[usenames]{xcolor}
\usepackage[nohead]{geometry}
\usepackage[doublespacing]{setspace}  %set square brackets to singlespacing or doublespacing
\usepackage[bottom]{footmisc}
\usepackage{indentfirst}
\usepackage[labelfont=bf]{caption}
\usepackage{authblk}
\usepackage{natbib}
\usepackage{subfig}
\usepackage[section]{placeins}
\usepackage[flushleft]{threeparttable}
\usepackage{endnotes}
\usepackage{rotating}
\usepackage{datetime}
\usepackage{bibentry}
\usepackage{wrapfig}
\usepackage{siunitx}
\usepackage{times}    %Times New Roman
\usepackage[urlcolor=blue,citecolor=blue,linkcolor=blue,colorlinks=true]{hyperref}
		%see http://en.wikibooks.org/wiki/LaTeX/Hyperlinks


\usepackage{dcolumn}  %decimal aligned columns
\newcolumntype{d}[1]{D{.}{\cdot}{#1} }
%the argument for d specifies the maximum number of decimal places

\newdateformat{mydate}{\monthname[\THEMONTH] \THEYEAR}

\geometry{left=1in,right=1in,top=1.00in,bottom=1.0in}
\marginsize{1 in}{1 in}{1 in}{1 in}
\setlength{\textfloatsep}{0.1cm}
\setlength{\floatsep}{1cm}
\renewcommand\floatpagefraction{.9}
\renewcommand\topfraction{.9}
\renewcommand\bottomfraction{.9}
\renewcommand\textfraction{.1}  

\renewcommand\Affilfont{\fontsize{10}{10.8}\itshape}   %changes font of affiliations
\newcommand{\jcomment}[2]{\hspace{0in}#2}  %custom, found on  web, to ignore everything in between, for commenting.



\title{EconS 581 Assignment 1: Discrete Choice and Recreational demand models}


\author[1]{Prof. Cook}


\date{\mydate\today}

\begin{document}

  \maketitle
\singlespace

\textbf{Question 16.3}

\textbf{A.} The ratio of the two coefficients is 1.36.  This is the marginal willingness to pay to reduce installation costs.  So a household is willing to pay \$136 to reduce installation costs by \$100.  This seems illogical but might be consistent with a present value calculation.  Imagine for example that household had to borrow money to finance installation of the system.  The present value of the installation would include those borrowing costs, or just the opportunity cost of capital. We can't say much more without knowing something about the useful life of the different heating options, and it does seem quite high to me.

\bigskip

\textbf{B.}  Given information in table, use formula 16.12 to calculate probabilities of choice.  Households are NOT assumed to simply make choices at random.  See spreadsheet on dropbox.
\bigskip

\textbf{D.}  The data in the table show the values for IC (in 100's of dollars), so calculate what a 15\% reduction would be for j=5.  Then multiply that by the marginal WTP from part a.  For person 3, 14 and 21, these are \$214, \$158 and \$221. See spreadsheet.
\bigskip



\textbf{Question 16.4}


\bigskip
\textbf{A.}   Column 1 of Table  \ref{tab:assgn1_tables16_4} reports the regression replicating the results in question 16.3.
\bigskip


\textbf{B.} Model 2 includes a set of alternate-specific constants, omitting the constant for j=1 (central gas).  These ASCs capture elements of the heating choices that are not captured by the operating or installation cost.  The coefficient on installation falls (less negative) and the operating cost increases (more negative).  MWTP is \$1.36 in the model without ASCs, and falls to \$0.22 in the model with them, or willing to pay \$22 in ongoing costs to reduce installation costs by \$100. These MWTPs are obviously driven by the changes in the ic and oc coefficients described above.  Most likely there are unobserved attributes of alternatives with high installation costs and low operating costs (or low installation and high operating) that were being conflated into the ic and oc parameter estimates.  Note that the ASCs are statistically different between 1 (central gas) and 2 (room gas heating) and 5 (heat pump).    


\singlespace
\begin{threeparttable}[h]  
\caption{Conditional logit models of home heating choices (question 16.4)}
{\label{tab:assgn1_tables16_4}}
\small
%\begin{center}
\input{assgn1_tables16_4.tex}
%\end{center}
\begin{tablenotes}
\footnotesize
\item \emph{Notes:} Standard errors in parentheses. *** significant at the 1\% level, **5\%, * 10\%.   \\
\end{tablenotes}
\end{threeparttable}
\doublespace


 \textbf{C.}  Models 3 and 4 test the impact of income, which must be interacted with an attribute that varies over alternatives to be included.  I interacted these with ASCs (see code) by defining one variable ($inc\_central$) that is equal to income if alt=1, 3 or 5, and zero otherwise.  Model 3 includes this interaction but not the ASCs, and model 4 includes the ASCs.  Are higher income households more likely to choose 1, 3 or 5?  Yes - the coefficient on $inc\_central$ is positive and significant.  The coefficient does not change much with the inclusion of the ASCs but it is now only marginally significant.

 \textbf{D.}  Using the predict command and manually reducing the installation cost by 15\% for j=5 changes the predicted probability of choosing j=5 from 5.56\% to 6.94\%.  See Stata code.


 \textbf{E.}  Here we are not just talking about marginal WTP for attribute changes, but total WTP for the product when one its attributes changes by 15\%.  This is akin to a quality change in P \& R, and we need to use the log-sum formula (16.54).  See my Stata code, but here is a place where I wish I had an answer myself!  I believe my welfare results indicate that households are willing to pay only \$0.46 for the 15\% rebate on installation costs, which seems pretty implausible to me. 




\textbf{Question 17.3a}

Column 1 of Table \ref{tab:assgn1_tables17_3a} shows the results of two count (Poisson) models of trips to one beach, Wrightsville Beach.  Income is not statistically significant determinant of trip demand, though travel cost is.  Using estimates from the model without income, I calculated the average welfare loss from eliminating the site to be \$70.22 (see code). I calculated the welfare loss from a \$5 increase in the access fee as \$65.97. (see code)




\singlespace
\begin{threeparttable}[h]  
\caption{Count (Poisson) regressions of the number of trips (tr10) to Wrightsville Beach}
{\label{tab:assgn1_tables17_3a}}
\small
%\begin{center}
\input{assgn1_tables17_3a.tex}
%\end{center}
\begin{tablenotes}
\footnotesize
\item \emph{Notes:} Standard errors in parentheses. *** significant at the 1\% level, **5\%, * 10\%.   \\
\end{tablenotes}
\end{threeparttable}
\doublespace

\textbf{Question 17.4}

\textbf{A}.   Model 1 in Table \ref{tab:assgn1_table17_4} reports the results of the conditional logit model explaining choice of the 100 angling sites.  Travel cost has a negative sign as expected, and catch rates for all three species increase the probability of choosing the site. By dividing the coefficient on walleye by the negative of the coefficient on travel cost, I calculated that respondents WTP is \$16.33 for increasing the expected walleye catch by 1 fish, recalling mean catch rate is 0.16 fish per trip.

\singlespace
\begin{threeparttable}[h]  
\caption{Conditional logit models of 100 angling destinations in Wisconsin}
{\label{tab:assgn1_table17_4}}
\small
%\begin{center}
\input{assgn1_table17_4.tex}
%\end{center}
\begin{tablenotes}
\footnotesize
\item \emph{Notes:} Standard errors in parentheses. *** significant at the 1\% level, **5\%, * 10\%.   Model 3 was estimated with 99 site-specific constants which are suppressed from output.\\
\end{tablenotes}
\end{threeparttable}
\doublespace


\textbf{B}.  Model 2 in Table \ref{tab:assgn1_table17_4} reports results with two other site-specific characteristics (ramp and restroom) and interactions with household-specific variables (own boat, and have children).  Respondents are less likely to choose a site with a ramp if they don't own a boat ($\beta _{ramp}$), but more likely if they have a boat ($\beta _{ramp_boat} - \beta _{ramp}$ is positive). Restrooms have a similar effect - they are only valued by families with children. Now respondents are WTP \$16.81 for a 1-unit increase in walleye catch rate.  This is slightly higher than our first estimate. The travel cost variable barely changed, but all three coefficients on catch rates increased. I believe this is because catch rates were positively correlated with other positively-valued site attributes, which may be particularly valued by families with children or visitors with boats.  For example, the pwcorr I ran above showed that all three catch rates are positively correlated with the presence of boat ramps and bathrooms, likely because the  sites with amenities are stocked or sites with good catch rates are highly visited and the state provides better amenities there.


\textbf{C}. I interpreted this question to mean increasing the quality at 10 sites, leaving the others the same, and use eq 17.27 to find the overall (all site) per-choice occasion WTP. The top 10 highest site visits were sites 33 22 14 19 91 18 16 58 60 62 61, 61.  I use the saved model results (Model 2) and just manually increase catch rates, and use equation 17.27 again. My calculation is that mean WTP for the policy increasing walleye catch rates at those ten sites is \$0.46 per trip. (See do file for code). The per choice occasion value of access for the most visited site (site 61) is \$0.22.  I operationalized this by setting the travel cost of site 61 to \$10,000 (see do file).


\textbf{D}. Model 3 in Table \ref{tab:assgn1_table17_4} estimates the first stage: a conditional logit model omitting site-level variables but including a series of site-specific dummies (ASCs).  They are suppressed from the output in the Table.  Table \ref{tab:assgn1_table17_4D} estimates the second-stage: an OLS model with the coefficients on those ASCs as the left-hand side, and site-level variables on the right-hand side.  First, the model fit actually improves (log likelihood gets smaller in the clogit stage even though we are adding 99-2 = 97 additional parameters to estimate:  there is significant unobserved site heterogeneity that is not well explained in models with site specific constants.  Second, I would note that in either model you can see that sites with restrooms and boat ramps seem to be less likely to be visited (i.e. negative coefficients in the second stage) but the interaction terms imply that for people with children and/or boats the net effect of the two coefficients in positive.  In the 2-stage model, however, the point estimate of ramp is still negative but not statistically significant.  I would guess these are wrapped up in crowding if sites more frequently visited are more likely to have bathrooms and boat ramps.  


\singlespace
\begin{threeparttable}[h]  
\caption{Second-stage OLS model explaining site-specific constants}
{\label{tab:assgn1_table17_4D}}
\small
%\begin{center}
\input{assgn1_table17_4D.tex}
%\end{center}
\begin{tablenotes}
\footnotesize
\item \emph{Notes:} Standard errors in parentheses. *** significant at the 1\% level, **5\%, * 10\%.   Model 3 was estimated with 99 site-specific constants which are suppressed from output.\\
\end{tablenotes}
\end{threeparttable}
\doublespace

\textbf{E}. I got a welfare gain for the increase in walleye catch rate of \$0.66 (compared to \$0.46) and  per-choice occasion welfare change from restricting access to site 61 of \$0.12, compared to \$0.68 above.  


\end{document}